
% Default to the notebook output style

    


% Inherit from the specified cell style.




    
\documentclass[11pt]{article}
\usepackage{xeCJK}

    
    
    \usepackage[T1]{fontenc}
    % Nicer default font (+ math font) than Computer Modern for most use cases
    \usepackage{mathpazo}

    % Basic figure setup, for now with no caption control since it's done
    % automatically by Pandoc (which extracts ![](path) syntax from Markdown).
    \usepackage{graphicx}
    % We will generate all images so they have a width \maxwidth. This means
    % that they will get their normal width if they fit onto the page, but
    % are scaled down if they would overflow the margins.
    \makeatletter
    \def\maxwidth{\ifdim\Gin@nat@width>\linewidth\linewidth
    \else\Gin@nat@width\fi}
    \makeatother
    \let\Oldincludegraphics\includegraphics
    % Set max figure width to be 80% of text width, for now hardcoded.
    \renewcommand{\includegraphics}[1]{\Oldincludegraphics[width=.8\maxwidth]{#1}}
    % Ensure that by default, figures have no caption (until we provide a
    % proper Figure object with a Caption API and a way to capture that
    % in the conversion process - todo).
    \usepackage{caption}
    \DeclareCaptionLabelFormat{nolabel}{}
    \captionsetup{labelformat=nolabel}

    \usepackage{adjustbox} % Used to constrain images to a maximum size 
    \usepackage{xcolor} % Allow colors to be defined
    \usepackage{enumerate} % Needed for markdown enumerations to work
    \usepackage{geometry} % Used to adjust the document margins
    \usepackage{amsmath} % Equations
    \usepackage{amssymb} % Equations
    \usepackage{textcomp} % defines textquotesingle
    % Hack from http://tex.stackexchange.com/a/47451/13684:
    \AtBeginDocument{%
        \def\PYZsq{\textquotesingle}% Upright quotes in Pygmentized code
    }
    \usepackage{upquote} % Upright quotes for verbatim code
    \usepackage{eurosym} % defines \euro
    \usepackage[mathletters]{ucs} % Extended unicode (utf-8) support
    \usepackage[utf8x]{inputenc} % Allow utf-8 characters in the tex document
    \usepackage{fancyvrb} % verbatim replacement that allows latex
    \usepackage{grffile} % extends the file name processing of package graphics 
                         % to support a larger range 
    % The hyperref package gives us a pdf with properly built
    % internal navigation ('pdf bookmarks' for the table of contents,
    % internal cross-reference links, web links for URLs, etc.)
    \usepackage{hyperref}
    \usepackage{longtable} % longtable support required by pandoc >1.10
    \usepackage{booktabs}  % table support for pandoc > 1.12.2
    \usepackage[inline]{enumitem} % IRkernel/repr support (it uses the enumerate* environment)
    \usepackage[normalem]{ulem} % ulem is needed to support strikethroughs (\sout)
                                % normalem makes italics be italics, not underlines
    

    
    
    % Colors for the hyperref package
    \definecolor{urlcolor}{rgb}{0,.145,.698}
    \definecolor{linkcolor}{rgb}{.71,0.21,0.01}
    \definecolor{citecolor}{rgb}{.12,.54,.11}

    % ANSI colors
    \definecolor{ansi-black}{HTML}{3E424D}
    \definecolor{ansi-black-intense}{HTML}{282C36}
    \definecolor{ansi-red}{HTML}{E75C58}
    \definecolor{ansi-red-intense}{HTML}{B22B31}
    \definecolor{ansi-green}{HTML}{00A250}
    \definecolor{ansi-green-intense}{HTML}{007427}
    \definecolor{ansi-yellow}{HTML}{DDB62B}
    \definecolor{ansi-yellow-intense}{HTML}{B27D12}
    \definecolor{ansi-blue}{HTML}{208FFB}
    \definecolor{ansi-blue-intense}{HTML}{0065CA}
    \definecolor{ansi-magenta}{HTML}{D160C4}
    \definecolor{ansi-magenta-intense}{HTML}{A03196}
    \definecolor{ansi-cyan}{HTML}{60C6C8}
    \definecolor{ansi-cyan-intense}{HTML}{258F8F}
    \definecolor{ansi-white}{HTML}{C5C1B4}
    \definecolor{ansi-white-intense}{HTML}{A1A6B2}

    % commands and environments needed by pandoc snippets
    % extracted from the output of `pandoc -s`
    \providecommand{\tightlist}{%
      \setlength{\itemsep}{0pt}\setlength{\parskip}{0pt}}
    \DefineVerbatimEnvironment{Highlighting}{Verbatim}{commandchars=\\\{\}}
    % Add ',fontsize=\small' for more characters per line
    \newenvironment{Shaded}{}{}
    \newcommand{\KeywordTok}[1]{\textcolor[rgb]{0.00,0.44,0.13}{\textbf{{#1}}}}
    \newcommand{\DataTypeTok}[1]{\textcolor[rgb]{0.56,0.13,0.00}{{#1}}}
    \newcommand{\DecValTok}[1]{\textcolor[rgb]{0.25,0.63,0.44}{{#1}}}
    \newcommand{\BaseNTok}[1]{\textcolor[rgb]{0.25,0.63,0.44}{{#1}}}
    \newcommand{\FloatTok}[1]{\textcolor[rgb]{0.25,0.63,0.44}{{#1}}}
    \newcommand{\CharTok}[1]{\textcolor[rgb]{0.25,0.44,0.63}{{#1}}}
    \newcommand{\StringTok}[1]{\textcolor[rgb]{0.25,0.44,0.63}{{#1}}}
    \newcommand{\CommentTok}[1]{\textcolor[rgb]{0.38,0.63,0.69}{\textit{{#1}}}}
    \newcommand{\OtherTok}[1]{\textcolor[rgb]{0.00,0.44,0.13}{{#1}}}
    \newcommand{\AlertTok}[1]{\textcolor[rgb]{1.00,0.00,0.00}{\textbf{{#1}}}}
    \newcommand{\FunctionTok}[1]{\textcolor[rgb]{0.02,0.16,0.49}{{#1}}}
    \newcommand{\RegionMarkerTok}[1]{{#1}}
    \newcommand{\ErrorTok}[1]{\textcolor[rgb]{1.00,0.00,0.00}{\textbf{{#1}}}}
    \newcommand{\NormalTok}[1]{{#1}}
    
    % Additional commands for more recent versions of Pandoc
    \newcommand{\ConstantTok}[1]{\textcolor[rgb]{0.53,0.00,0.00}{{#1}}}
    \newcommand{\SpecialCharTok}[1]{\textcolor[rgb]{0.25,0.44,0.63}{{#1}}}
    \newcommand{\VerbatimStringTok}[1]{\textcolor[rgb]{0.25,0.44,0.63}{{#1}}}
    \newcommand{\SpecialStringTok}[1]{\textcolor[rgb]{0.73,0.40,0.53}{{#1}}}
    \newcommand{\ImportTok}[1]{{#1}}
    \newcommand{\DocumentationTok}[1]{\textcolor[rgb]{0.73,0.13,0.13}{\textit{{#1}}}}
    \newcommand{\AnnotationTok}[1]{\textcolor[rgb]{0.38,0.63,0.69}{\textbf{\textit{{#1}}}}}
    \newcommand{\CommentVarTok}[1]{\textcolor[rgb]{0.38,0.63,0.69}{\textbf{\textit{{#1}}}}}
    \newcommand{\VariableTok}[1]{\textcolor[rgb]{0.10,0.09,0.49}{{#1}}}
    \newcommand{\ControlFlowTok}[1]{\textcolor[rgb]{0.00,0.44,0.13}{\textbf{{#1}}}}
    \newcommand{\OperatorTok}[1]{\textcolor[rgb]{0.40,0.40,0.40}{{#1}}}
    \newcommand{\BuiltInTok}[1]{{#1}}
    \newcommand{\ExtensionTok}[1]{{#1}}
    \newcommand{\PreprocessorTok}[1]{\textcolor[rgb]{0.74,0.48,0.00}{{#1}}}
    \newcommand{\AttributeTok}[1]{\textcolor[rgb]{0.49,0.56,0.16}{{#1}}}
    \newcommand{\InformationTok}[1]{\textcolor[rgb]{0.38,0.63,0.69}{\textbf{\textit{{#1}}}}}
    \newcommand{\WarningTok}[1]{\textcolor[rgb]{0.38,0.63,0.69}{\textbf{\textit{{#1}}}}}
    
    
    % Define a nice break command that doesn't care if a line doesn't already
    % exist.
    \def\br{\hspace*{\fill} \\* }
    % Math Jax compatability definitions
    \def\gt{>}
    \def\lt{<}
    % Document parameters
    \title{ctr-prediction-demo}
    
    
    

    % Pygments definitions
    
\makeatletter
\def\PY@reset{\let\PY@it=\relax \let\PY@bf=\relax%
    \let\PY@ul=\relax \let\PY@tc=\relax%
    \let\PY@bc=\relax \let\PY@ff=\relax}
\def\PY@tok#1{\csname PY@tok@#1\endcsname}
\def\PY@toks#1+{\ifx\relax#1\empty\else%
    \PY@tok{#1}\expandafter\PY@toks\fi}
\def\PY@do#1{\PY@bc{\PY@tc{\PY@ul{%
    \PY@it{\PY@bf{\PY@ff{#1}}}}}}}
\def\PY#1#2{\PY@reset\PY@toks#1+\relax+\PY@do{#2}}

\expandafter\def\csname PY@tok@w\endcsname{\def\PY@tc##1{\textcolor[rgb]{0.73,0.73,0.73}{##1}}}
\expandafter\def\csname PY@tok@c\endcsname{\let\PY@it=\textit\def\PY@tc##1{\textcolor[rgb]{0.25,0.50,0.50}{##1}}}
\expandafter\def\csname PY@tok@cp\endcsname{\def\PY@tc##1{\textcolor[rgb]{0.74,0.48,0.00}{##1}}}
\expandafter\def\csname PY@tok@k\endcsname{\let\PY@bf=\textbf\def\PY@tc##1{\textcolor[rgb]{0.00,0.50,0.00}{##1}}}
\expandafter\def\csname PY@tok@kp\endcsname{\def\PY@tc##1{\textcolor[rgb]{0.00,0.50,0.00}{##1}}}
\expandafter\def\csname PY@tok@kt\endcsname{\def\PY@tc##1{\textcolor[rgb]{0.69,0.00,0.25}{##1}}}
\expandafter\def\csname PY@tok@o\endcsname{\def\PY@tc##1{\textcolor[rgb]{0.40,0.40,0.40}{##1}}}
\expandafter\def\csname PY@tok@ow\endcsname{\let\PY@bf=\textbf\def\PY@tc##1{\textcolor[rgb]{0.67,0.13,1.00}{##1}}}
\expandafter\def\csname PY@tok@nb\endcsname{\def\PY@tc##1{\textcolor[rgb]{0.00,0.50,0.00}{##1}}}
\expandafter\def\csname PY@tok@nf\endcsname{\def\PY@tc##1{\textcolor[rgb]{0.00,0.00,1.00}{##1}}}
\expandafter\def\csname PY@tok@nc\endcsname{\let\PY@bf=\textbf\def\PY@tc##1{\textcolor[rgb]{0.00,0.00,1.00}{##1}}}
\expandafter\def\csname PY@tok@nn\endcsname{\let\PY@bf=\textbf\def\PY@tc##1{\textcolor[rgb]{0.00,0.00,1.00}{##1}}}
\expandafter\def\csname PY@tok@ne\endcsname{\let\PY@bf=\textbf\def\PY@tc##1{\textcolor[rgb]{0.82,0.25,0.23}{##1}}}
\expandafter\def\csname PY@tok@nv\endcsname{\def\PY@tc##1{\textcolor[rgb]{0.10,0.09,0.49}{##1}}}
\expandafter\def\csname PY@tok@no\endcsname{\def\PY@tc##1{\textcolor[rgb]{0.53,0.00,0.00}{##1}}}
\expandafter\def\csname PY@tok@nl\endcsname{\def\PY@tc##1{\textcolor[rgb]{0.63,0.63,0.00}{##1}}}
\expandafter\def\csname PY@tok@ni\endcsname{\let\PY@bf=\textbf\def\PY@tc##1{\textcolor[rgb]{0.60,0.60,0.60}{##1}}}
\expandafter\def\csname PY@tok@na\endcsname{\def\PY@tc##1{\textcolor[rgb]{0.49,0.56,0.16}{##1}}}
\expandafter\def\csname PY@tok@nt\endcsname{\let\PY@bf=\textbf\def\PY@tc##1{\textcolor[rgb]{0.00,0.50,0.00}{##1}}}
\expandafter\def\csname PY@tok@nd\endcsname{\def\PY@tc##1{\textcolor[rgb]{0.67,0.13,1.00}{##1}}}
\expandafter\def\csname PY@tok@s\endcsname{\def\PY@tc##1{\textcolor[rgb]{0.73,0.13,0.13}{##1}}}
\expandafter\def\csname PY@tok@sd\endcsname{\let\PY@it=\textit\def\PY@tc##1{\textcolor[rgb]{0.73,0.13,0.13}{##1}}}
\expandafter\def\csname PY@tok@si\endcsname{\let\PY@bf=\textbf\def\PY@tc##1{\textcolor[rgb]{0.73,0.40,0.53}{##1}}}
\expandafter\def\csname PY@tok@se\endcsname{\let\PY@bf=\textbf\def\PY@tc##1{\textcolor[rgb]{0.73,0.40,0.13}{##1}}}
\expandafter\def\csname PY@tok@sr\endcsname{\def\PY@tc##1{\textcolor[rgb]{0.73,0.40,0.53}{##1}}}
\expandafter\def\csname PY@tok@ss\endcsname{\def\PY@tc##1{\textcolor[rgb]{0.10,0.09,0.49}{##1}}}
\expandafter\def\csname PY@tok@sx\endcsname{\def\PY@tc##1{\textcolor[rgb]{0.00,0.50,0.00}{##1}}}
\expandafter\def\csname PY@tok@m\endcsname{\def\PY@tc##1{\textcolor[rgb]{0.40,0.40,0.40}{##1}}}
\expandafter\def\csname PY@tok@gh\endcsname{\let\PY@bf=\textbf\def\PY@tc##1{\textcolor[rgb]{0.00,0.00,0.50}{##1}}}
\expandafter\def\csname PY@tok@gu\endcsname{\let\PY@bf=\textbf\def\PY@tc##1{\textcolor[rgb]{0.50,0.00,0.50}{##1}}}
\expandafter\def\csname PY@tok@gd\endcsname{\def\PY@tc##1{\textcolor[rgb]{0.63,0.00,0.00}{##1}}}
\expandafter\def\csname PY@tok@gi\endcsname{\def\PY@tc##1{\textcolor[rgb]{0.00,0.63,0.00}{##1}}}
\expandafter\def\csname PY@tok@gr\endcsname{\def\PY@tc##1{\textcolor[rgb]{1.00,0.00,0.00}{##1}}}
\expandafter\def\csname PY@tok@ge\endcsname{\let\PY@it=\textit}
\expandafter\def\csname PY@tok@gs\endcsname{\let\PY@bf=\textbf}
\expandafter\def\csname PY@tok@gp\endcsname{\let\PY@bf=\textbf\def\PY@tc##1{\textcolor[rgb]{0.00,0.00,0.50}{##1}}}
\expandafter\def\csname PY@tok@go\endcsname{\def\PY@tc##1{\textcolor[rgb]{0.53,0.53,0.53}{##1}}}
\expandafter\def\csname PY@tok@gt\endcsname{\def\PY@tc##1{\textcolor[rgb]{0.00,0.27,0.87}{##1}}}
\expandafter\def\csname PY@tok@err\endcsname{\def\PY@bc##1{\setlength{\fboxsep}{0pt}\fcolorbox[rgb]{1.00,0.00,0.00}{1,1,1}{\strut ##1}}}
\expandafter\def\csname PY@tok@kc\endcsname{\let\PY@bf=\textbf\def\PY@tc##1{\textcolor[rgb]{0.00,0.50,0.00}{##1}}}
\expandafter\def\csname PY@tok@kd\endcsname{\let\PY@bf=\textbf\def\PY@tc##1{\textcolor[rgb]{0.00,0.50,0.00}{##1}}}
\expandafter\def\csname PY@tok@kn\endcsname{\let\PY@bf=\textbf\def\PY@tc##1{\textcolor[rgb]{0.00,0.50,0.00}{##1}}}
\expandafter\def\csname PY@tok@kr\endcsname{\let\PY@bf=\textbf\def\PY@tc##1{\textcolor[rgb]{0.00,0.50,0.00}{##1}}}
\expandafter\def\csname PY@tok@bp\endcsname{\def\PY@tc##1{\textcolor[rgb]{0.00,0.50,0.00}{##1}}}
\expandafter\def\csname PY@tok@fm\endcsname{\def\PY@tc##1{\textcolor[rgb]{0.00,0.00,1.00}{##1}}}
\expandafter\def\csname PY@tok@vc\endcsname{\def\PY@tc##1{\textcolor[rgb]{0.10,0.09,0.49}{##1}}}
\expandafter\def\csname PY@tok@vg\endcsname{\def\PY@tc##1{\textcolor[rgb]{0.10,0.09,0.49}{##1}}}
\expandafter\def\csname PY@tok@vi\endcsname{\def\PY@tc##1{\textcolor[rgb]{0.10,0.09,0.49}{##1}}}
\expandafter\def\csname PY@tok@vm\endcsname{\def\PY@tc##1{\textcolor[rgb]{0.10,0.09,0.49}{##1}}}
\expandafter\def\csname PY@tok@sa\endcsname{\def\PY@tc##1{\textcolor[rgb]{0.73,0.13,0.13}{##1}}}
\expandafter\def\csname PY@tok@sb\endcsname{\def\PY@tc##1{\textcolor[rgb]{0.73,0.13,0.13}{##1}}}
\expandafter\def\csname PY@tok@sc\endcsname{\def\PY@tc##1{\textcolor[rgb]{0.73,0.13,0.13}{##1}}}
\expandafter\def\csname PY@tok@dl\endcsname{\def\PY@tc##1{\textcolor[rgb]{0.73,0.13,0.13}{##1}}}
\expandafter\def\csname PY@tok@s2\endcsname{\def\PY@tc##1{\textcolor[rgb]{0.73,0.13,0.13}{##1}}}
\expandafter\def\csname PY@tok@sh\endcsname{\def\PY@tc##1{\textcolor[rgb]{0.73,0.13,0.13}{##1}}}
\expandafter\def\csname PY@tok@s1\endcsname{\def\PY@tc##1{\textcolor[rgb]{0.73,0.13,0.13}{##1}}}
\expandafter\def\csname PY@tok@mb\endcsname{\def\PY@tc##1{\textcolor[rgb]{0.40,0.40,0.40}{##1}}}
\expandafter\def\csname PY@tok@mf\endcsname{\def\PY@tc##1{\textcolor[rgb]{0.40,0.40,0.40}{##1}}}
\expandafter\def\csname PY@tok@mh\endcsname{\def\PY@tc##1{\textcolor[rgb]{0.40,0.40,0.40}{##1}}}
\expandafter\def\csname PY@tok@mi\endcsname{\def\PY@tc##1{\textcolor[rgb]{0.40,0.40,0.40}{##1}}}
\expandafter\def\csname PY@tok@il\endcsname{\def\PY@tc##1{\textcolor[rgb]{0.40,0.40,0.40}{##1}}}
\expandafter\def\csname PY@tok@mo\endcsname{\def\PY@tc##1{\textcolor[rgb]{0.40,0.40,0.40}{##1}}}
\expandafter\def\csname PY@tok@ch\endcsname{\let\PY@it=\textit\def\PY@tc##1{\textcolor[rgb]{0.25,0.50,0.50}{##1}}}
\expandafter\def\csname PY@tok@cm\endcsname{\let\PY@it=\textit\def\PY@tc##1{\textcolor[rgb]{0.25,0.50,0.50}{##1}}}
\expandafter\def\csname PY@tok@cpf\endcsname{\let\PY@it=\textit\def\PY@tc##1{\textcolor[rgb]{0.25,0.50,0.50}{##1}}}
\expandafter\def\csname PY@tok@c1\endcsname{\let\PY@it=\textit\def\PY@tc##1{\textcolor[rgb]{0.25,0.50,0.50}{##1}}}
\expandafter\def\csname PY@tok@cs\endcsname{\let\PY@it=\textit\def\PY@tc##1{\textcolor[rgb]{0.25,0.50,0.50}{##1}}}

\def\PYZbs{\char`\\}
\def\PYZus{\char`\_}
\def\PYZob{\char`\{}
\def\PYZcb{\char`\}}
\def\PYZca{\char`\^}
\def\PYZam{\char`\&}
\def\PYZlt{\char`\<}
\def\PYZgt{\char`\>}
\def\PYZsh{\char`\#}
\def\PYZpc{\char`\%}
\def\PYZdl{\char`\$}
\def\PYZhy{\char`\-}
\def\PYZsq{\char`\'}
\def\PYZdq{\char`\"}
\def\PYZti{\char`\~}
% for compatibility with earlier versions
\def\PYZat{@}
\def\PYZlb{[}
\def\PYZrb{]}
\makeatother


    % Exact colors from NB
    \definecolor{incolor}{rgb}{0.0, 0.0, 0.5}
    \definecolor{outcolor}{rgb}{0.545, 0.0, 0.0}



    
    % Prevent overflowing lines due to hard-to-break entities
    \sloppy 
    % Setup hyperref package
    \hypersetup{
      breaklinks=true,  % so long urls are correctly broken across lines
      colorlinks=true,
      urlcolor=urlcolor,
      linkcolor=linkcolor,
      citecolor=citecolor,
      }
    % Slightly bigger margins than the latex defaults
    
    \geometry{verbose,tmargin=1in,bmargin=1in,lmargin=1in,rmargin=1in}
    
    

    \begin{document}
    
    
    \maketitle
    
    

    
    项目地址:\url{https://github.com/georgethrax/ctr-prediction-demo}

下载项目文件后,在本地浏览器中打开 \texttt{ctr-prediction-demo.html}
以查看本文档。

    \section{问题描述}\label{ux95eeux9898ux63cfux8ff0}

\begin{itemize}
\item
  问题背景:2015在线广告点击率(CTR)预估大赛
  https://www.kaggle.com/c/avazu-ctr-prediction
\item
  任务目标:根据广告的特征数据,预测一个广告是否被用户点击(点击/未点击的二分类问题)
\item
  数据文件:\texttt{ctr\_data.csv}。原始数据过大,这里截取10000条数据。
\item
  数据字段:

  \begin{itemize}
  \tightlist
  \item
    id
  \item
    click 是否点击,0/1
  \item
    hour
  \item
    C1 一个个类别型特征(categorical feature),具体业务含义被隐去
  \item
    banner\_pos
  \item
    site\_id
  \item
    site\_domain
  \item
    site\_category
  \item
    app\_id
  \item
    app\_domain
  \item
    app\_category
  \item
    device\_id
  \item
    device\_ip
  \item
    device\_model
  \item
    device\_type
  \item
    device\_conn\_type
  \item
    C14-C21 一些类别型特征
  \end{itemize}
\end{itemize}

其中,id不使用,click 被作为标签,其他字段可以被用作特征

    \section{环境配置}\label{ux73afux5883ux914dux7f6e}

    \subsection{安装Anaconda3:python3发行版}\label{ux5b89ux88c5anaconda3python3ux53d1ux884cux7248}

安装Anaconda3后,本文所用到的python库就已经包含在内了。

从 https://www.anaconda.com/distribution/ 下载安装包并安装即可。

    \subsection{运行代码}\label{ux8fd0ux884cux4ee3ux7801}

从GitHub下载本项目https://github.com/georgethrax/ctr-prediction-demo
后,有以下几种方式来运行代码:

    \subsubsection{通过jupyter
notebook运行代码(推荐的方式)}\label{ux901aux8fc7jupyter-notebookux8fd0ux884cux4ee3ux7801ux63a8ux8350ux7684ux65b9ux5f0f}

\paragraph{打开控制台(Windows CMD,Linux/MacOS
Terminal),跳转到本项目文件所在的目录}\label{ux6253ux5f00ux63a7ux5236ux53f0windows-cmdlinuxmacos-terminalux8df3ux8f6cux5230ux672cux9879ux76eeux6587ux4ef6ux6240ux5728ux7684ux76eeux5f55}

\begin{verbatim}
cd ctr-prediction-demo
\end{verbatim}

\paragraph{启动jupyter notebook}\label{ux542fux52a8jupyter-notebook}

\begin{verbatim}
jupyter notebook
\end{verbatim}

此时会自动浏览器

\paragraph{\texorpdfstring{打开本项目中的\texttt{ctr-prediction-demo.ipynb}文件,按顺序执行代码即可}{打开本项目中的ctr-prediction-demo.ipynb文件,按顺序执行代码即可}}\label{ux6253ux5f00ux672cux9879ux76eeux4e2dux7684ctr-prediction-demo.ipynbux6587ux4ef6ux6309ux987aux5e8fux6267ux884cux4ee3ux7801ux5373ux53ef}

    \subsubsection{通过spyder运行代码}\label{ux901aux8fc7spyderux8fd0ux884cux4ee3ux7801}

spyder是随Anaconda安装好的一个轻量级python
IDE。用spyder打开\texttt{ctr\_prediction-demo.py}并运行即可。

    \subsubsection{直接在控制台运行代码}\label{ux76f4ux63a5ux5728ux63a7ux5236ux53f0ux8fd0ux884cux4ee3ux7801}

\paragraph{打开控制台(Windows CMD,Linux/MacOS
Terminal),跳转到本项目文件所在的目录}\label{ux6253ux5f00ux63a7ux5236ux53f0windows-cmdlinuxmacos-terminalux8df3ux8f6cux5230ux672cux9879ux76eeux6587ux4ef6ux6240ux5728ux7684ux76eeux5f55}

\begin{verbatim}
cd ctr-prediction-demo
\end{verbatim}

\paragraph{执行代码}\label{ux6267ux884cux4ee3ux7801}

\begin{verbatim}
python ctr-prediction-demo.py
\end{verbatim}

    \section{收集数据}\label{ux6536ux96c6ux6570ux636e}

    这里假设数据已经收集并整理为磁盘文件ctr\_data.csv

    \begin{Verbatim}[commandchars=\\\{\}]
{\color{incolor}In [{\color{incolor}1}]:} \PY{k+kn}{import} \PY{n+nn}{pandas} \PY{k}{as} \PY{n+nn}{pd}
        \PY{k+kn}{from} \PY{n+nn}{sklearn}\PY{n+nn}{.}\PY{n+nn}{preprocessing} \PY{k}{import} \PY{n}{OneHotEncoder}\PY{p}{,} \PY{n}{LabelEncoder}
        \PY{k+kn}{from} \PY{n+nn}{sklearn}\PY{n+nn}{.}\PY{n+nn}{tree} \PY{k}{import} \PY{n}{DecisionTreeClassifier}
        \PY{k+kn}{from} \PY{n+nn}{sklearn}\PY{n+nn}{.}\PY{n+nn}{cross\PYZus{}validation} \PY{k}{import} \PY{n}{train\PYZus{}test\PYZus{}split}
        \PY{k+kn}{from} \PY{n+nn}{sklearn}\PY{n+nn}{.}\PY{n+nn}{metrics} \PY{k}{import} \PY{n}{accuracy\PYZus{}score}\PY{p}{,} \PY{n}{roc\PYZus{}auc\PYZus{}score}\PY{p}{,} \PY{n}{log\PYZus{}loss}
        \PY{k+kn}{import} \PY{n+nn}{warnings}
        \PY{n}{warnings}\PY{o}{.}\PY{n}{simplefilter}\PY{p}{(}\PY{l+s+s2}{\PYZdq{}}\PY{l+s+s2}{ignore}\PY{l+s+s2}{\PYZdq{}}\PY{p}{)}
\end{Verbatim}


    \begin{Verbatim}[commandchars=\\\{\}]
/home/lix/anaconda3/lib/python3.6/site-packages/sklearn/cross\_validation.py:41: DeprecationWarning: This module was deprecated in version 0.18 in favor of the model\_selection module into which all the refactored classes and functions are moved. Also note that the interface of the new CV iterators are different from that of this module. This module will be removed in 0.20.
  "This module will be removed in 0.20.", DeprecationWarning)

    \end{Verbatim}

    \begin{Verbatim}[commandchars=\\\{\}]
{\color{incolor}In [{\color{incolor}2}]:} \PY{c+c1}{\PYZsh{} 读取数据集}
        \PY{n}{df} \PY{o}{=} \PY{n}{pd}\PY{o}{.}\PY{n}{read\PYZus{}csv}\PY{p}{(}\PY{l+s+s2}{\PYZdq{}}\PY{l+s+s2}{./ctr\PYZus{}data.csv}\PY{l+s+s2}{\PYZdq{}}\PY{p}{,} \PY{n}{index\PYZus{}col}\PY{o}{=}\PY{k+kc}{None}\PY{p}{)}
        \PY{n}{df}\PY{o}{.}\PY{n}{head}\PY{p}{(}\PY{p}{)}
\end{Verbatim}


\begin{Verbatim}[commandchars=\\\{\}]
{\color{outcolor}Out[{\color{outcolor}2}]:}              id  click      hour    C1  banner\_pos   site\_id site\_domain  \textbackslash{}
        0  1.000009e+18      0  14102100  1005           0  1fbe01fe    f3845767   
        1  1.000017e+19      0  14102100  1005           0  1fbe01fe    f3845767   
        2  1.000037e+19      0  14102100  1005           0  1fbe01fe    f3845767   
        3  1.000064e+19      0  14102100  1005           0  1fbe01fe    f3845767   
        4  1.000068e+19      0  14102100  1005           1  fe8cc448    9166c161   
        
          site\_category    app\_id app\_domain  {\ldots} device\_type device\_conn\_type    C14  \textbackslash{}
        0      28905ebd  ecad2386   7801e8d9  {\ldots}           1                2  15706   
        1      28905ebd  ecad2386   7801e8d9  {\ldots}           1                0  15704   
        2      28905ebd  ecad2386   7801e8d9  {\ldots}           1                0  15704   
        3      28905ebd  ecad2386   7801e8d9  {\ldots}           1                0  15706   
        4      0569f928  ecad2386   7801e8d9  {\ldots}           1                0  18993   
        
           C15  C16   C17  C18  C19     C20  C21  
        0  320   50  1722    0   35      -1   79  
        1  320   50  1722    0   35  100084   79  
        2  320   50  1722    0   35  100084   79  
        3  320   50  1722    0   35  100084   79  
        4  320   50  2161    0   35      -1  157  
        
        [5 rows x 24 columns]
\end{Verbatim}
            
    \section{特征工程}\label{ux7279ux5f81ux5de5ux7a0b}

为简单起见,这里仅考虑特征选择和类别型特征编码。

实际场景中,可能面临缺失值处理、离群点处理、日期型特征编码、数据降维等等。

    \subsection{特征选择}\label{ux7279ux5f81ux9009ux62e9}

设置用到的字段/特征/列

    \begin{Verbatim}[commandchars=\\\{\}]
{\color{incolor}In [{\color{incolor}3}]:} \PY{n}{cols\PYZus{}data} \PY{o}{=} \PY{p}{[}\PY{l+s+s1}{\PYZsq{}}\PY{l+s+s1}{C1}\PY{l+s+s1}{\PYZsq{}}\PY{p}{,}\PY{l+s+s1}{\PYZsq{}}\PY{l+s+s1}{banner\PYZus{}pos}\PY{l+s+s1}{\PYZsq{}}\PY{p}{,} \PY{l+s+s1}{\PYZsq{}}\PY{l+s+s1}{site\PYZus{}domain}\PY{l+s+s1}{\PYZsq{}}\PY{p}{,}  \PY{l+s+s1}{\PYZsq{}}\PY{l+s+s1}{site\PYZus{}id}\PY{l+s+s1}{\PYZsq{}}\PY{p}{,} \PY{l+s+s1}{\PYZsq{}}\PY{l+s+s1}{site\PYZus{}category}\PY{l+s+s1}{\PYZsq{}}\PY{p}{,}\PY{l+s+s1}{\PYZsq{}}\PY{l+s+s1}{app\PYZus{}id}\PY{l+s+s1}{\PYZsq{}}\PY{p}{,}\PYZbs{}
                     \PY{l+s+s1}{\PYZsq{}}\PY{l+s+s1}{app\PYZus{}category}\PY{l+s+s1}{\PYZsq{}}\PY{p}{,}  \PY{l+s+s1}{\PYZsq{}}\PY{l+s+s1}{device\PYZus{}type}\PY{l+s+s1}{\PYZsq{}}\PY{p}{,}  \PY{l+s+s1}{\PYZsq{}}\PY{l+s+s1}{device\PYZus{}conn\PYZus{}type}\PY{l+s+s1}{\PYZsq{}}\PY{p}{,} \PY{l+s+s1}{\PYZsq{}}\PY{l+s+s1}{C14}\PY{l+s+s1}{\PYZsq{}}\PY{p}{,} \PY{l+s+s1}{\PYZsq{}}\PY{l+s+s1}{C15}\PY{l+s+s1}{\PYZsq{}}\PY{p}{,}\PY{l+s+s1}{\PYZsq{}}\PY{l+s+s1}{C16}\PY{l+s+s1}{\PYZsq{}}\PY{p}{]}
        \PY{n}{cols\PYZus{}label} \PY{o}{=} \PY{p}{[}\PY{l+s+s1}{\PYZsq{}}\PY{l+s+s1}{click}\PY{l+s+s1}{\PYZsq{}}\PY{p}{]}
\end{Verbatim}


    由设置好的特征字段,构造数据集X和标签y

    \begin{Verbatim}[commandchars=\\\{\}]
{\color{incolor}In [{\color{incolor}4}]:} \PY{n}{X} \PY{o}{=} \PY{n}{df}\PY{p}{[}\PY{n}{cols\PYZus{}data}\PY{p}{]} 
        \PY{n}{y} \PY{o}{=} \PY{n}{df}\PY{p}{[}\PY{n}{cols\PYZus{}label}\PY{p}{]}  
\end{Verbatim}


    \subsection{特征编码}\label{ux7279ux5f81ux7f16ux7801}

特征编码:将原始数据的字符串等特征转换为模型能够处理的数值型特征。LR,SVM类模型可以使用OneHotEncoder。决策树类模型可以使用LabelEncoder。

为简单起见,本文仅讨论决策树类模型,故仅使用LabelEncoder特征编码

    \begin{Verbatim}[commandchars=\\\{\}]
{\color{incolor}In [{\color{incolor}5}]:} \PY{n}{cols\PYZus{}categorical} \PY{o}{=} \PY{p}{[}\PY{l+s+s1}{\PYZsq{}}\PY{l+s+s1}{site\PYZus{}domain}\PY{l+s+s1}{\PYZsq{}}\PY{p}{,} \PY{l+s+s1}{\PYZsq{}}\PY{l+s+s1}{site\PYZus{}id}\PY{l+s+s1}{\PYZsq{}}\PY{p}{,} \PY{l+s+s1}{\PYZsq{}}\PY{l+s+s1}{site\PYZus{}category}\PY{l+s+s1}{\PYZsq{}}\PY{p}{,} \PY{l+s+s1}{\PYZsq{}}\PY{l+s+s1}{app\PYZus{}id}\PY{l+s+s1}{\PYZsq{}}\PY{p}{,} \PYZbs{}
                            \PY{l+s+s1}{\PYZsq{}}\PY{l+s+s1}{app\PYZus{}category}\PY{l+s+s1}{\PYZsq{}}\PY{p}{]}
        \PY{n}{lbl} \PY{o}{=} \PY{n}{LabelEncoder}\PY{p}{(}\PY{p}{)}
        
        \PY{k}{for} \PY{n}{col} \PY{o+ow}{in} \PY{n}{cols\PYZus{}categorical}\PY{p}{:}
            \PY{n+nb}{print}\PY{p}{(}\PY{n}{col}\PY{p}{)}
            \PY{n}{X}\PY{p}{[}\PY{n}{col}\PY{p}{]} \PY{o}{=} \PY{n}{lbl}\PY{o}{.}\PY{n}{fit\PYZus{}transform}\PY{p}{(}\PY{n}{X}\PY{p}{[}\PY{n}{col}\PY{p}{]}\PY{p}{)}
\end{Verbatim}


    \begin{Verbatim}[commandchars=\\\{\}]
site\_domain
site\_id
site\_category
app\_id
app\_category

    \end{Verbatim}

    \begin{Verbatim}[commandchars=\\\{\}]
{\color{incolor}In [{\color{incolor}6}]:} \PY{n}{X}\PY{o}{.}\PY{n}{head}\PY{p}{(}\PY{p}{)}
\end{Verbatim}


\begin{Verbatim}[commandchars=\\\{\}]
{\color{outcolor}Out[{\color{outcolor}6}]:}      C1  banner\_pos  site\_domain  site\_id  site\_category  app\_id  \textbackslash{}
        0  1005           0          301       43              2     293   
        1  1005           0          301       43              2     293   
        2  1005           0          301       43              2     293   
        3  1005           0          301       43              2     293   
        4  1005           1          169      374              0     293   
        
           app\_category  device\_type  device\_conn\_type    C14  C15  C16  
        0             0            1                 2  15706  320   50  
        1             0            1                 0  15704  320   50  
        2             0            1                 0  15704  320   50  
        3             0            1                 0  15706  320   50  
        4             0            1                 0  18993  320   50  
\end{Verbatim}
            
    \subsection{划分训练集、测试集}\label{ux5212ux5206ux8badux7ec3ux96c6ux6d4bux8bd5ux96c6}

这里采用训练集占80\%,测试集占20\%

    \begin{Verbatim}[commandchars=\\\{\}]
{\color{incolor}In [{\color{incolor}7}]:} \PY{n}{X\PYZus{}train}\PY{p}{,}\PY{n}{X\PYZus{}test}\PY{p}{,} \PY{n}{y\PYZus{}train}\PY{p}{,} \PY{n}{y\PYZus{}test} \PY{o}{=}  \PYZbs{}
            \PY{n}{train\PYZus{}test\PYZus{}split}\PY{p}{(}\PY{n}{X}\PY{p}{,} \PY{n}{y}\PY{p}{,} \PY{n}{test\PYZus{}size}\PY{o}{=}\PY{l+m+mf}{0.2}\PY{p}{,} \PY{n}{random\PYZus{}state}\PY{o}{=}\PY{l+m+mi}{0}\PY{p}{)}
\end{Verbatim}


    \section{建立一个模型,并训练、测试}\label{ux5efaux7acbux4e00ux4e2aux6a21ux578bux5e76ux8badux7ec3ux6d4bux8bd5}

    这里调用一个sklearn算法库中现成的决策树分类器DecisionTreeClassifier,记为clf1

    \subsection{创建模型}\label{ux521bux5efaux6a21ux578b}

创建一个分类模型,命名为clf1,使用默认模型参数

    \begin{Verbatim}[commandchars=\\\{\}]
{\color{incolor}In [{\color{incolor}8}]:} \PY{n}{clf1} \PY{o}{=} \PY{n}{DecisionTreeClassifier}\PY{p}{(}\PY{p}{)}
\end{Verbatim}


    \subsection{训练}\label{ux8badux7ec3}

在训练集上训练分类器clf1

    \begin{Verbatim}[commandchars=\\\{\}]
{\color{incolor}In [{\color{incolor}9}]:} \PY{n}{clf1}\PY{o}{.}\PY{n}{fit}\PY{p}{(}\PY{n}{X\PYZus{}train}\PY{p}{,} \PY{n}{y\PYZus{}train}\PY{p}{)}
\end{Verbatim}


\begin{Verbatim}[commandchars=\\\{\}]
{\color{outcolor}Out[{\color{outcolor}9}]:} DecisionTreeClassifier(class\_weight=None, criterion='gini', max\_depth=None,
                    max\_features=None, max\_leaf\_nodes=None,
                    min\_impurity\_decrease=0.0, min\_impurity\_split=None,
                    min\_samples\_leaf=1, min\_samples\_split=2,
                    min\_weight\_fraction\_leaf=0.0, presort=False, random\_state=None,
                    splitter='best')
\end{Verbatim}
            
    \subsection{预测}\label{ux9884ux6d4b}

使用训练好的分类器clf1,在测试集上预测分类结果

预测结果有两种形式: - y\_score: 为每个测试样本 x
预测一个0.0\textasciitilde{}1.0的实数,表示 x 被分类为类别1的概率 -
y\_pred: 为每个测试样本 x 预测一个0/1类别标签。当 y\_score(x)
\textgreater{} 0.5 时,y\_pred(x) = 1。当 y\_score(x) \textless{} 0.5
时,y\_pred(x) = 0。

    \begin{Verbatim}[commandchars=\\\{\}]
{\color{incolor}In [{\color{incolor}10}]:} \PY{c+c1}{\PYZsh{}分类器预测的类别为1的概率值/分数值}
         \PY{n}{y\PYZus{}score} \PY{o}{=} \PY{n}{clf1}\PY{o}{.}\PY{n}{predict\PYZus{}proba}\PY{p}{(}\PY{n}{X\PYZus{}test}\PY{p}{)}\PY{p}{[}\PY{p}{:}\PY{p}{,} \PY{n}{clf1}\PY{o}{.}\PY{n}{classes\PYZus{}} \PY{o}{==} \PY{l+m+mi}{1}\PY{p}{]} 
         
         \PY{c+c1}{\PYZsh{}按阈值(默认0.5)将y\PYZus{}score二值化为0/1预测标签}
         \PY{n}{y\PYZus{}pred} \PY{o}{=} \PY{n}{clf1}\PY{o}{.}\PY{n}{predict}\PY{p}{(}\PY{n}{X\PYZus{}test}\PY{p}{)}      
\end{Verbatim}


    \subsection{评估}\label{ux8bc4ux4f30}

评估预测结果,使用ACC, AUC, logloss等评价指标。ACC,
AUC越接近于1,logloss越小,分类效果越好。

    \begin{Verbatim}[commandchars=\\\{\}]
{\color{incolor}In [{\color{incolor}11}]:} \PY{n}{acc} \PY{o}{=} \PY{n}{accuracy\PYZus{}score}\PY{p}{(}\PY{n}{y\PYZus{}test}\PY{p}{,} \PY{n}{y\PYZus{}pred}\PY{p}{)}
         \PY{n}{auc} \PY{o}{=} \PY{n}{roc\PYZus{}auc\PYZus{}score}\PY{p}{(}\PY{n}{y\PYZus{}test}\PY{p}{,} \PY{n}{y\PYZus{}score}\PY{p}{)}
         \PY{n}{logloss} \PY{o}{=} \PY{n}{log\PYZus{}loss}\PY{p}{(}\PY{n}{y\PYZus{}test}\PY{p}{,} \PY{n}{y\PYZus{}score}\PY{p}{)}
         \PY{n+nb}{print}\PY{p}{(}\PY{n}{acc}\PY{p}{,} \PY{n}{auc}\PY{p}{,} \PY{n}{logloss}\PY{p}{)}
\end{Verbatim}


    \begin{Verbatim}[commandchars=\\\{\}]
0.822 0.6741693349061904 1.9120927413144937

    \end{Verbatim}

    \section{修改模型参数,重新训练、测试}\label{ux4feeux6539ux6a21ux578bux53c2ux6570ux91cdux65b0ux8badux7ec3ux6d4bux8bd5}

为模型clf1换一组参数,记为clf1\_p1

出于演示目的,不妨令clf1\_p1中的一个模型参数修改为
max\_leaf\_nodes=10。(clf1原参数为max\_leaf\_nodes=None)

    \subsection{创建模型}\label{ux521bux5efaux6a21ux578b}

    \begin{Verbatim}[commandchars=\\\{\}]
{\color{incolor}In [{\color{incolor}12}]:} \PY{n}{clf1\PYZus{}p1} \PY{o}{=} \PY{n}{DecisionTreeClassifier}\PY{p}{(} \PY{n}{max\PYZus{}leaf\PYZus{}nodes}\PY{o}{=}\PY{l+m+mi}{10}\PY{p}{)}
\end{Verbatim}


    \subsection{训练}\label{ux8badux7ec3}

    \begin{Verbatim}[commandchars=\\\{\}]
{\color{incolor}In [{\color{incolor}13}]:} \PY{n}{clf1\PYZus{}p1}\PY{o}{.}\PY{n}{fit}\PY{p}{(}\PY{n}{X\PYZus{}train}\PY{p}{,} \PY{n}{y\PYZus{}train}\PY{p}{)}
\end{Verbatim}


\begin{Verbatim}[commandchars=\\\{\}]
{\color{outcolor}Out[{\color{outcolor}13}]:} DecisionTreeClassifier(class\_weight=None, criterion='gini', max\_depth=None,
                     max\_features=None, max\_leaf\_nodes=10,
                     min\_impurity\_decrease=0.0, min\_impurity\_split=None,
                     min\_samples\_leaf=1, min\_samples\_split=2,
                     min\_weight\_fraction\_leaf=0.0, presort=False, random\_state=None,
                     splitter='best')
\end{Verbatim}
            
    \subsection{预测}\label{ux9884ux6d4b}

    \begin{Verbatim}[commandchars=\\\{\}]
{\color{incolor}In [{\color{incolor}14}]:} \PY{n}{y\PYZus{}score} \PY{o}{=} \PY{n}{clf1\PYZus{}p1}\PY{o}{.}\PY{n}{predict\PYZus{}proba}\PY{p}{(}\PY{n}{X\PYZus{}test}\PY{p}{)}\PY{p}{[}\PY{p}{:}\PY{p}{,} \PY{n}{clf1\PYZus{}p1}\PY{o}{.}\PY{n}{classes\PYZus{}} \PY{o}{==} \PY{l+m+mi}{1}\PY{p}{]} 
         \PY{n}{y\PYZus{}pred} \PY{o}{=} \PY{n}{clf1\PYZus{}p1}\PY{o}{.}\PY{n}{predict}\PY{p}{(}\PY{n}{X\PYZus{}test}\PY{p}{)} 
\end{Verbatim}


    \subsection{评估}\label{ux8bc4ux4f30}

    \begin{Verbatim}[commandchars=\\\{\}]
{\color{incolor}In [{\color{incolor}15}]:} \PY{n}{acc} \PY{o}{=} \PY{n}{accuracy\PYZus{}score}\PY{p}{(}\PY{n}{y\PYZus{}test}\PY{p}{,} \PY{n}{y\PYZus{}pred}\PY{p}{)}
         \PY{n}{auc} \PY{o}{=} \PY{n}{roc\PYZus{}auc\PYZus{}score}\PY{p}{(}\PY{n}{y\PYZus{}test}\PY{p}{,} \PY{n}{y\PYZus{}score}\PY{p}{)}
         \PY{n}{logloss} \PY{o}{=} \PY{n}{log\PYZus{}loss}\PY{p}{(}\PY{n}{y\PYZus{}test}\PY{p}{,} \PY{n}{y\PYZus{}score}\PY{p}{)}
         \PY{n+nb}{print}\PY{p}{(}\PY{n}{acc}\PY{p}{,} \PY{n}{auc}\PY{p}{,} \PY{n}{logloss}\PY{p}{)}
\end{Verbatim}


    \begin{Verbatim}[commandchars=\\\{\}]
0.828 0.6583099862375015 0.43256841278416175

    \end{Verbatim}

    \textbf{从评估指标来看,模型clf1\_p1比clf1差。}

    \section{更换模型,重新训练、测试}\label{ux66f4ux6362ux6a21ux578bux91cdux65b0ux8badux7ec3ux6d4bux8bd5}

这里换一个sklearn库中现成的GradientBoostingClassifier,记为clf2

    \begin{Verbatim}[commandchars=\\\{\}]
{\color{incolor}In [{\color{incolor}16}]:} \PY{k+kn}{from} \PY{n+nn}{sklearn}\PY{n+nn}{.}\PY{n+nn}{ensemble} \PY{k}{import} \PY{n}{GradientBoostingClassifier}
         \PY{n}{clf2} \PY{o}{=} \PY{n}{GradientBoostingClassifier}\PY{p}{(}\PY{p}{)} 
\end{Verbatim}


    \subsection{训练}\label{ux8badux7ec3}

    \begin{Verbatim}[commandchars=\\\{\}]
{\color{incolor}In [{\color{incolor}17}]:} \PY{n}{clf2}\PY{o}{.}\PY{n}{fit}\PY{p}{(}\PY{n}{X\PYZus{}train}\PY{p}{,} \PY{n}{y\PYZus{}train}\PY{p}{)} 
\end{Verbatim}


\begin{Verbatim}[commandchars=\\\{\}]
{\color{outcolor}Out[{\color{outcolor}17}]:} GradientBoostingClassifier(criterion='friedman\_mse', init=None,
                       learning\_rate=0.1, loss='deviance', max\_depth=3,
                       max\_features=None, max\_leaf\_nodes=None,
                       min\_impurity\_decrease=0.0, min\_impurity\_split=None,
                       min\_samples\_leaf=1, min\_samples\_split=2,
                       min\_weight\_fraction\_leaf=0.0, n\_estimators=100,
                       presort='auto', random\_state=None, subsample=1.0, verbose=0,
                       warm\_start=False)
\end{Verbatim}
            
    \subsection{预测}\label{ux9884ux6d4b}

    \begin{Verbatim}[commandchars=\\\{\}]
{\color{incolor}In [{\color{incolor}18}]:} \PY{n}{y\PYZus{}score} \PY{o}{=} \PY{n}{clf2}\PY{o}{.}\PY{n}{predict\PYZus{}proba}\PY{p}{(}\PY{n}{X\PYZus{}test}\PY{p}{)}\PY{p}{[}\PY{p}{:}\PY{p}{,} \PY{n}{clf2}\PY{o}{.}\PY{n}{classes\PYZus{}} \PY{o}{==} \PY{l+m+mi}{1}\PY{p}{]} 
         \PY{n}{y\PYZus{}pred} \PY{o}{=} \PY{n}{clf2}\PY{o}{.}\PY{n}{predict}\PY{p}{(}\PY{n}{X\PYZus{}test}\PY{p}{)} 
\end{Verbatim}


    \subsection{评估}\label{ux8bc4ux4f30}

    \begin{Verbatim}[commandchars=\\\{\}]
{\color{incolor}In [{\color{incolor}19}]:} \PY{n}{acc} \PY{o}{=} \PY{n}{accuracy\PYZus{}score}\PY{p}{(}\PY{n}{y\PYZus{}test}\PY{p}{,} \PY{n}{y\PYZus{}pred}\PY{p}{)}
         \PY{n}{auc} \PY{o}{=} \PY{n}{roc\PYZus{}auc\PYZus{}score}\PY{p}{(}\PY{n}{y\PYZus{}test}\PY{p}{,} \PY{n}{y\PYZus{}score}\PY{p}{)}
         \PY{n}{logloss} \PY{o}{=} \PY{n}{log\PYZus{}loss}\PY{p}{(}\PY{n}{y\PYZus{}test}\PY{p}{,} \PY{n}{y\PYZus{}score}\PY{p}{)}
         \PY{n+nb}{print}\PY{p}{(}\PY{n}{acc}\PY{p}{,} \PY{n}{auc}\PY{p}{,} \PY{n}{logloss}\PY{p}{)}
\end{Verbatim}


    \begin{Verbatim}[commandchars=\\\{\}]
0.8225 0.6870286695315133 0.4252470403545188

    \end{Verbatim}

    \textbf{从测试集上的评估指标来看,模型clf2比clf1,clf1\_p1好}

    \section{模型迭代}\label{ux6a21ux578bux8fedux4ee3}

将收集数据、特征工程、模型选择、模型参数选择、训练测试等步骤反复迭代,直到评价指标令人满意为止。


    % Add a bibliography block to the postdoc
    
    
    
    \end{document}
